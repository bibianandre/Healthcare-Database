\documentclass[12pt,a4paper]{article}
\usepackage[legalpaper, portrait, margin=2cm]{geometry}
\usepackage{fancyhdr}
\usepackage{amsmath}
\usepackage{amssymb}
\usepackage{graphicx}
\usepackage{wrapfig}
\usepackage{blindtext}
\usepackage{hyperref}
\usepackage{enumitem}
\usepackage{pdflscape}
\usepackage{svg}
\usepackage{listings}
\usepackage{xcolor}

\graphicspath{ {./} }
\hypersetup{
  colorlinks=true,
  linkcolor=blue,
  filecolor=magenta,
  urlcolor=blue,
  citecolor=blue,
  pdftitle={Projeto BD - Parte 1 2023/2024 LEIC-A},
  pdfpagemode=FullScreen,
}

\pagestyle{fancy}
\fancyhf{}
\rhead{Grupo \textbf{49}}
\lhead{Projeto BD (Parte 1) 2023/2024 LEIC-A}
\cfoot{Bibiana André (94158) Ricardo Henriques (106228) Afonso Jacinto (106458)}

\renewcommand{\footrulewidth}{0.2pt}

\renewcommand{\labelitemii}{$\circ$}
\renewcommand{\labelitemiii}{$\diamond$}

\begin{document}
  \begin{titlepage}
    \begin{center}
      \vspace*{5cm}
      \Huge
      \textbf{Projeto BD - Parte 1}

      \vspace{0.5cm}
      \LARGE
      Grupo 49 | Turno L08 | LEIC-A

      \vspace{1.0cm}
      \large
      Prof.ª Daniela Machado

      \vspace{2.0cm}
      \begin{figure}[h]
          \centering
          \includegraphics[scale=0.3]{IST_Logo.png}
      \end{figure}

      \vfill
      \large
      \begin{minipage}{0.8\textwidth}
        \begin{itemize}
          \item[] \textbf{Bibiana André} (ist194158) - 33.3$\overline{3}$\% - 28h
          \item[] \textbf{Ricardo Henriques} (ist1106228) - 33.3$\overline{3}$\% - 28h
          \item[] \textbf{Afonso Jacinto} (ist1106458) - 33.3$\overline{3}$\% - 28h
        \end{itemize}
      \end{minipage}
    \end{center}
  \end{titlepage}

\begin{figure}[htbp]
    \centering
    \hspace*{-2cm}
    \includegraphics[scale=0.95]{diagrama_page-0001.jpg}
\end{figure}

\section*{Modelo Entidade-Associação}
  \subsection*{Restrições de Integridade}
  \footnotesize
  \begin{itemize}
    \item \textbf{(RI-1)} O número de telefone é \textbf{único}.
    \item \textbf{(RI-2)} O número de NIF é \textbf{único}.
    \item \textbf{(RI-3)} "Dia da semana" tem como atributo "nome" apenas nomes de \textbf{dias úteis}.
    \item \textbf{(RI-4)} O atributo "hora" da entidade "Marcação" está compreendido entre as 8.00h e as 20.00h e corresponde a um \textbf{período de 30 minutos}.
    \item \textbf{(RI-5)}  Cada par "Marcação" - "Médico" tem \textbf{no máximo} um "Paciente".
    \item \textbf{(RI-6)} Cada par "Marcação" - "Paciente" tem \textbf{no máximo} um "Médico".
    \item \textbf{(RI-7)} O atributo "data" da entidade "Marcação" corresponde a um dia no formato DD/MM/YYYY de calendário.
    \item \textbf{(RI-8)} O atributo "nome" da entidade "Medicamento" tem de constar na lista oficial do INFARMED.
    \item \textbf{(RI-9)} O atributo "nome" da entidade "Sintoma qualitativo" tem de constar na lista SNOMED CT.
    \item \textbf{(RI-10)} O atributo "especialidade" da entidade "Médico" tem de constar na lista de especialidades reconhecida pela Ordem dos Médicos.
    \item \textbf{(RI-11)} Cada par "Médico" - "Dia da semana" correponde \textbf{apenas a uma} "Clínica".
    \item \textbf{(RI-12)} Para cada "Médico" que atende um "Paciente", os NIF do "Médico" e do "Paciente" apresentam \textbf{valores diferentes}.
    \item \textbf{(RI-13)} O atributo "morada" é \textbf{único}.
  \end{itemize}

  \subsection*{Justificações de Desenho}
  \normalsize
  \begin{itemize}
    \item De \textit{"Os pacientes podem
ter um ou mais sistemas de saúde (e.g. ADSE, AdvanceCare), identificados pelo nome do sistema e pelo
número de sistema nacional de saúde do paciente."}, interpretou-se que "Sistema de saúde", com chave fraca "nome", é uma entidade fraca de "Paciente", no sentido em que o número de SNS é a chave do paciente e o sistema de saúde tem nome único quando associado a um paciente.
    \item Uma consulta consiste numa agregação que associa uma entidade "Paciente" a uma entidade  "Médico" através de uma "Marcação" (com hora e data). A "Clínica" onde decorrerá a consulta é a clínica a que o "Médico" se encontra associado no "Dia da semana" a que corresponde a "Marcação".
    \item Optou-se por omitir do diagrama a entidade "Serviços de Saúde", que consistem em consultas (dadas por médicos) e serviços de enfermagem (prestados por enfermeiros), pois a consulta consiste numa agregacão entre "Paciente", "Médico" e "Marcação" e os serviços de enfermagem (igualmente omitidos) não têm manifestação relevante ao nível dos dados.
    \item Interpretou-se que os sintomas qualitativos registados na consulta são resultado da descrição feita pelo paciente (relação "descreve") e os sintomas quantitativos registados na consulta são resultado da observação feita pelo médico (relação "observa").
    \item Decidiu-se não incluir nada quanto à imutabilidade de certos atributos por não estar explicitada na descrição do domínio como é o caso de "telefone" e "morada" (exceto da entidade "Clínica", cuja unicidade é explicitada) e "IBAN" da entidade "Profissional de saúde".
    \item Considerando que um "Médico" também pode ser um "Paciente" em contexto de consulta caso se verifique uma igualdade no valor do atributo "NIF", no diagrama modelou-se que um "Médico" está impedido de se atender a si próprio numa consulta.
     
  \end{itemize}
  
\newpage
\section*{Modelo Relacional}
  \ttfamily
  \noindent
  H(\underline{h1}, h2)

  \vspace*{10pt}
  \noindent
  E(\underline{e1},\underline{e2})

  \vspace*{10pt}
  \noindent
  A(\underline{a1}, a2, a3)

  \vspace*{10pt}
  \noindent
  B(\underline{a1}, b1)
  \begin{itemize}[nosep]
      \item[]a1: FK(A)
  \end{itemize}

  \vspace*{10pt}
  \noindent
  C(\underline{a1})
  \begin{itemize}[nosep]
      \item[]a1: FK(A)
  \end{itemize}

  \vspace*{10pt}
  \noindent
  G(\underline{g1})

  \vspace*{10pt}
  \noindent
  F(\underline{f1}, \underline{f2}, f3)

  \vspace*{10pt}
  \noindent
  rCE(\underline{a1}, e1, e2, rce1)
  \begin{itemize}[nosep]
      \item[]a1: FK(C)
      \item[]e1, e2: FK(E.e1, E.e2) NOT NULL
  \end{itemize}

  \vspace*{10pt}
  \noindent
  rAFG(\underline{f1}, \underline{f2}, \underline{a1}, \underline{g1}, h1)
  \begin{itemize}[nosep]
      \item[]f1, f2: FK(F.f1, F.f2)
      \item[]a1: FK(A)
      \item[]g1: FK(G)
      \item[]h1: FK(H) NOT NULL 
      \item[] unique(f1, f2, a1)
      \item[]\textsf{\textbf{(RI-2)}}:todos os f1, f2 de F têm de estar em rAFG
      \item[]\textsf{\textbf{(RI-3)}}:todos os h1 de H têm de estar em rAFG
  \end{itemize}

  \vspace*{10pt}
  \noindent
  D(\underline{d1}, \underline{f1}, \underline{f2}, \underline{a1}, \underline{g1})
  \begin{itemize}[nosep]
      \item[]f1, f2, a1, g1: FK(rAFG)
  \end{itemize}
\rmfamily
\vspace*{20pt}
\noindent


  Foi mantida a numeração das Restrições de Integridade consoante as disponibilizadas no enunciado, iniciando a contagem a 2 para as RI adicionadas. A RI-1 disponibilizada no modelo original foi convertida para o modelo relacional através da propriedade \ttfamily "unique". \rmfamily

\newpage

\section*{Álgebra Relacional {\&} SQL}

  \begin{enumerate}
    \item \textbf{Qual a expressão de álgebra relacional \underline{mais concisa} para a interrogação “quais os pacientes que consultaram médicos de todas as especialidades”?}
    \[
      \begin{aligned}
        & \Pi _{\; \op{p.nome, p.SSN, m.especialidade} \;}( \rho _{\op{p}} (paciente) \bowtie _{\; \op{p.SSN = c.SSN}} \rho _{\op{c}} (consulta) \bowtie _{\; \op{c.NIF = m.NIF}} \rho _{\op{m}} (medico)) \\
        & \div \\
        & \Pi _{\op{especialidade}}(medico)
      \end{aligned}
    \]

    \item \textbf{Indique a interrogação em linguagem natural a que corresponde a seguinte expressão de
álgebra relacional:}
    \[
      \begin{aligned}
        & r \; \leftarrow \; _{\op{especialidade} \;} G _{\; \op{count()} \; \mapsto \; \op{consultas} \;}(\op{consulta} \bowtie _{\; \op{consulta.NIF = medico.NIF}} \op{medico}) \\
        & \Pi _{\; \op{especialidade} \;}(medico) - \Pi _{\op{r1.especialidade}} (\sigma _{\; \op{r1.consultas \; < \; \op{r2.consultas}}} ( \rho _{\op{r1}} (r) \times \rho _{\op{r2}} (r)})
      \end{aligned}
    \]

    \underline{Resposta:} Qual a especialidade ou especialidades que tiveram o maior número de consultas?

    \item \textbf{Indique a \underline{interrogação em linguagem natural} a que corresponde a seguinte expressão de SQL:}
    
    \ttfamily
    
    \vspace*{10pt}
    \noindent
    SELECT p.SSN, p.nome
    
    \vspace*{1pt}
    \noindent
    FROM paciente p JOIN consulta c ON p.SSN = c.SSN

    \vspace*{1pt}
    \noindent
    GROUP BY p.SSN, p.nome, DATE(c.periodo)

    \vspace*{1pt}
    \noindent
    HAVING COUNT(*) > 1;
    \rmfamily

    \vspace{10pt}
    \noindent
    \underline{Resposta:} Qual o SSN e nome dos pacientes que foram a mais do que uma consulta no mesmo dia?

    \item \textbf{Comentário à expressão SQL apresentada pelo ChatGPT:}

    Na expressão SQL apresentada, relevaram-se os seguintes erros e imprecisões:
    \begin{itemize}
        \item Na 2ª e 3ª linhas da query, as expressões \ttfamily "COUNT(c.periodo) AS consultas\textunderscore medico" \rmfamily e \ttfamily "COUNT(DISTINCT p.NIF) AS total\textunderscore pacientes" \rmfamily atribuem os valores de \ttfamily COUNT() \rmfamily a \ttfamily "consultas\textunderscore medico" \rmfamily e \ttfamily "total\textunderscore pacientes" \rmfamily, para uso posterior. No entanto, abaixo, nas linhas subsequentes, essas expressões não são utilizadas, voltando-se a calcular desnecessariamente os valores antes calculados.


        \item De seguida, o cálculo da proporção de fidelidade não está de acordo com o pretendido. A proporção de fidelidade poderia definir-se como, por exemplo, o número total de consultas que os pacientes de um determinado médico tiveram com outros médicos da mesma especialidade, a dividir pelo número de total de consultas que esses mesmos pacientes tiveram com o médico em questão: quanto menor o rácio, mais fiéis são os pacientes desse médico. Ou, então, simplesmente, o número \textbf{total} de consultas que os pacientes de um dado médico tiveram no âmbito da sua especialidade, a dividir pelo número total de consultas dadas por esse médico dessa especialidade (novamente quanto menor o rácio, maior o grau de fidelidade dos pacientes do médico em questão). No entanto, o rácio apresentado na expressão não só não tem em conta a especialidade do médico em momento algum, como ainda consiste no número total de consultas registadas no sistema a dividir pelo número total de pacientes, o que nada nos diz acerca de cada médico, nem aparenta fazer sentido ou ter alguma utilidade, neste contexto.
        \item Posteriormente, a cláusula \ttfamily "WHERE"\rmfamily abaixo contém a instrução \ttfamily "SELECT MIN(periodo)", \rmfamily pelo que esta interrogação apenas terá em conta a consulta, ou as consultas, mais antigas da tabela de consultas, porque está a selecionar aquela cujo período é o menor, ou mais antigo. Isto não é o que se pretende, pois queremos ter em conta o total de consultas a que um paciente foi e não apenas a(s) que se localizam mais atrás no tempo.
        \item Por último, na penúltima linha, a expressão \ttfamily "GROUP BY m.nome" \rmfamily está a agrupar os resultados através dos nomes dos médicos. No entanto, esta abordagem não parece ser a mais adequada, pois pode haver mais do que um médico com o mesmo nome. Seria correto agrupar também (ou apenas) pelo atributo "NIF", que é chave primária da entidade "médico" e, como tal, identifica univocamente cada instância de "médico", não deixando margem para ambiguidades. % juntamente com o nome do paciente (para facilitar a interpretação do resultado).
    \end{itemize}
    
  \end{enumerate}

\end{document}
